\documentclass[12pt]{article}
\usepackage[margin=1in]{geometry}
\usepackage{graphicx}
\graphicspath{ {c:/John/LaTeX/Scioly/Wright Stuff/Torque Burners/} } % Be sure to choose the correct file path. 
\usepackage{grffile}

\author{John Yang}
\date{March 2019}
\title{Torque Burners}

\begin{document}
\maketitle

Torque burners are an idea originally introduced by Louis Garami (Feb. 1977 INAV), further developed by Dick Obarski (May 1989 INAV). 

Imagine you are flying an indoor free-flight model, and a portion of your rubber comes into a knot while winding. During the flight, the knot comes loose, and all of a sudden, there is more torque and the plane starts to climb for a bit. The torque burner uses this idea and tries to recreate it intentionally. This techinique is best suited for rooms with low celings, but can be implimented anywhere. Generally, in rooms with high celings, it is better to focus on other parameters that can still produce 2-3 minute flights with Wright Stuff parameters. 
\begin{figure}[h]
\caption{Louis Garami's first design}
\centering
\includegraphics[scale=0.65]{"principle"}
\end{figure}

The earliest design by Louis Garami (fig. 1) involved an S-hook and two short motors. The motors were wound together with the S-hook in between. A pin is mount to the motorstick and prevents the hook from turning. Eventually, when the first motor is unwound, the still wound back motor pulls on it enough so that it releases the hook and the second motor becomes engaged. This provides enough torque for the plane to climb again, but not as high. 

Obarski tried this approach and realized that the S-hook would often become stuck in the pin, and the second motor would never unwind. Eventually, he designed an 80\% effective model. 
\begin{figure}[h]
\caption{Obarski's design}
\centering
\includegraphics[scale=0.6]{"drawings"}
\end{figure}

It involves a homemade bushing made with plastic, and 0.007" or 0.009" music wire which can be obtained at any music shop or can be ordered online. The bushing is tied to the rubber using thin thread (0.005/0.006"), and the bushing goes over the wire during a flight. This idea was implimented on an Easy-B pennyplane, built at around 3 grams. One of the biggest concerns was that the jolt of the rubber coming loose would damage the airplane or disturb the flight. However, with Wright Stuff parameters, this should not be an issue. 

Another similar arrangement was created by Brett Sanborn, a member of the 2016 US F1D team. F1D planes are built at a minimum of 1.4 grams with only 0.4 grams of rubber, and are capable of achieving flight times of up to 40 minutes. His model can be easily implimented in a Wright Stuff plane. His idea was similar to Obarski's except it did not involve a bushing or a special motor. A similar arrangement of wires was used on the motorstick. However, instead of a modified motor, plastic tubes or straws were placed on the motor during winding. When fully wound, the straws were placed over the wires and removed. The principle is the same, however, it doesn't require manufacturing of a custom piece. 

Torque burners are a great way of increasing flight time in low celings, but it requires extensive testing and development to fit an individual's own aircraft. While it would be a great experiment to do, it is still important nonetheless to trim an aircraft properly, focusing on factors of the aircraft itself. For rooms with high celings it is better to perfect factors such as rubber, propeller pitch, wing angle, wing location, stablizer position and angle. 

\end{document}